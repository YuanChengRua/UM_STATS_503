% Options for packages loaded elsewhere
\PassOptionsToPackage{unicode}{hyperref}
\PassOptionsToPackage{hyphens}{url}
%
\documentclass[
]{article}
\usepackage{lmodern}
\usepackage{amssymb,amsmath}
\usepackage{ifxetex,ifluatex}
\ifnum 0\ifxetex 1\fi\ifluatex 1\fi=0 % if pdftex
  \usepackage[T1]{fontenc}
  \usepackage[utf8]{inputenc}
  \usepackage{textcomp} % provide euro and other symbols
\else % if luatex or xetex
  \usepackage{unicode-math}
  \defaultfontfeatures{Scale=MatchLowercase}
  \defaultfontfeatures[\rmfamily]{Ligatures=TeX,Scale=1}
\fi
% Use upquote if available, for straight quotes in verbatim environments
\IfFileExists{upquote.sty}{\usepackage{upquote}}{}
\IfFileExists{microtype.sty}{% use microtype if available
  \usepackage[]{microtype}
  \UseMicrotypeSet[protrusion]{basicmath} % disable protrusion for tt fonts
}{}
\makeatletter
\@ifundefined{KOMAClassName}{% if non-KOMA class
  \IfFileExists{parskip.sty}{%
    \usepackage{parskip}
  }{% else
    \setlength{\parindent}{0pt}
    \setlength{\parskip}{6pt plus 2pt minus 1pt}}
}{% if KOMA class
  \KOMAoptions{parskip=half}}
\makeatother
\usepackage{xcolor}
\IfFileExists{xurl.sty}{\usepackage{xurl}}{} % add URL line breaks if available
\IfFileExists{bookmark.sty}{\usepackage{bookmark}}{\usepackage{hyperref}}
\hypersetup{
  pdftitle={Q4},
  hidelinks,
  pdfcreator={LaTeX via pandoc}}
\urlstyle{same} % disable monospaced font for URLs
\usepackage[margin=1in]{geometry}
\usepackage{color}
\usepackage{fancyvrb}
\newcommand{\VerbBar}{|}
\newcommand{\VERB}{\Verb[commandchars=\\\{\}]}
\DefineVerbatimEnvironment{Highlighting}{Verbatim}{commandchars=\\\{\}}
% Add ',fontsize=\small' for more characters per line
\usepackage{framed}
\definecolor{shadecolor}{RGB}{248,248,248}
\newenvironment{Shaded}{\begin{snugshade}}{\end{snugshade}}
\newcommand{\AlertTok}[1]{\textcolor[rgb]{0.94,0.16,0.16}{#1}}
\newcommand{\AnnotationTok}[1]{\textcolor[rgb]{0.56,0.35,0.01}{\textbf{\textit{#1}}}}
\newcommand{\AttributeTok}[1]{\textcolor[rgb]{0.77,0.63,0.00}{#1}}
\newcommand{\BaseNTok}[1]{\textcolor[rgb]{0.00,0.00,0.81}{#1}}
\newcommand{\BuiltInTok}[1]{#1}
\newcommand{\CharTok}[1]{\textcolor[rgb]{0.31,0.60,0.02}{#1}}
\newcommand{\CommentTok}[1]{\textcolor[rgb]{0.56,0.35,0.01}{\textit{#1}}}
\newcommand{\CommentVarTok}[1]{\textcolor[rgb]{0.56,0.35,0.01}{\textbf{\textit{#1}}}}
\newcommand{\ConstantTok}[1]{\textcolor[rgb]{0.00,0.00,0.00}{#1}}
\newcommand{\ControlFlowTok}[1]{\textcolor[rgb]{0.13,0.29,0.53}{\textbf{#1}}}
\newcommand{\DataTypeTok}[1]{\textcolor[rgb]{0.13,0.29,0.53}{#1}}
\newcommand{\DecValTok}[1]{\textcolor[rgb]{0.00,0.00,0.81}{#1}}
\newcommand{\DocumentationTok}[1]{\textcolor[rgb]{0.56,0.35,0.01}{\textbf{\textit{#1}}}}
\newcommand{\ErrorTok}[1]{\textcolor[rgb]{0.64,0.00,0.00}{\textbf{#1}}}
\newcommand{\ExtensionTok}[1]{#1}
\newcommand{\FloatTok}[1]{\textcolor[rgb]{0.00,0.00,0.81}{#1}}
\newcommand{\FunctionTok}[1]{\textcolor[rgb]{0.00,0.00,0.00}{#1}}
\newcommand{\ImportTok}[1]{#1}
\newcommand{\InformationTok}[1]{\textcolor[rgb]{0.56,0.35,0.01}{\textbf{\textit{#1}}}}
\newcommand{\KeywordTok}[1]{\textcolor[rgb]{0.13,0.29,0.53}{\textbf{#1}}}
\newcommand{\NormalTok}[1]{#1}
\newcommand{\OperatorTok}[1]{\textcolor[rgb]{0.81,0.36,0.00}{\textbf{#1}}}
\newcommand{\OtherTok}[1]{\textcolor[rgb]{0.56,0.35,0.01}{#1}}
\newcommand{\PreprocessorTok}[1]{\textcolor[rgb]{0.56,0.35,0.01}{\textit{#1}}}
\newcommand{\RegionMarkerTok}[1]{#1}
\newcommand{\SpecialCharTok}[1]{\textcolor[rgb]{0.00,0.00,0.00}{#1}}
\newcommand{\SpecialStringTok}[1]{\textcolor[rgb]{0.31,0.60,0.02}{#1}}
\newcommand{\StringTok}[1]{\textcolor[rgb]{0.31,0.60,0.02}{#1}}
\newcommand{\VariableTok}[1]{\textcolor[rgb]{0.00,0.00,0.00}{#1}}
\newcommand{\VerbatimStringTok}[1]{\textcolor[rgb]{0.31,0.60,0.02}{#1}}
\newcommand{\WarningTok}[1]{\textcolor[rgb]{0.56,0.35,0.01}{\textbf{\textit{#1}}}}
\usepackage{graphicx,grffile}
\makeatletter
\def\maxwidth{\ifdim\Gin@nat@width>\linewidth\linewidth\else\Gin@nat@width\fi}
\def\maxheight{\ifdim\Gin@nat@height>\textheight\textheight\else\Gin@nat@height\fi}
\makeatother
% Scale images if necessary, so that they will not overflow the page
% margins by default, and it is still possible to overwrite the defaults
% using explicit options in \includegraphics[width, height, ...]{}
\setkeys{Gin}{width=\maxwidth,height=\maxheight,keepaspectratio}
% Set default figure placement to htbp
\makeatletter
\def\fps@figure{htbp}
\makeatother
\setlength{\emergencystretch}{3em} % prevent overfull lines
\providecommand{\tightlist}{%
  \setlength{\itemsep}{0pt}\setlength{\parskip}{0pt}}
\setcounter{secnumdepth}{-\maxdimen} % remove section numbering

\title{Q4}
\author{}
\date{\vspace{-2.5em}}

\begin{document}
\maketitle

\begin{Shaded}
\begin{Highlighting}[]
\NormalTok{train_data =}\StringTok{ }\KeywordTok{read.csv}\NormalTok{(}\StringTok{"E:}\CharTok{\textbackslash{}\textbackslash{}}\StringTok{diabetes_train.csv"}\NormalTok{)}
\NormalTok{test_data =}\StringTok{ }\KeywordTok{read.csv}\NormalTok{(}\StringTok{"E:}\CharTok{\textbackslash{}\textbackslash{}}\StringTok{diabetes_test.csv"}\NormalTok{)}

\KeywordTok{library}\NormalTok{(funModeling) }
\end{Highlighting}
\end{Shaded}

\begin{verbatim}
## Warning: package 'funModeling' was built under R version 4.0.3
\end{verbatim}

\begin{verbatim}
## Loading required package: Hmisc
\end{verbatim}

\begin{verbatim}
## Warning: package 'Hmisc' was built under R version 4.0.3
\end{verbatim}

\begin{verbatim}
## Loading required package: lattice
\end{verbatim}

\begin{verbatim}
## Loading required package: survival
\end{verbatim}

\begin{verbatim}
## Loading required package: Formula
\end{verbatim}

\begin{verbatim}
## Warning: package 'Formula' was built under R version 4.0.3
\end{verbatim}

\begin{verbatim}
## Loading required package: ggplot2
\end{verbatim}

\begin{verbatim}
## Warning: package 'ggplot2' was built under R version 4.0.3
\end{verbatim}

\begin{verbatim}
## 
## Attaching package: 'Hmisc'
\end{verbatim}

\begin{verbatim}
## The following objects are masked from 'package:base':
## 
##     format.pval, units
\end{verbatim}

\begin{verbatim}
## funModeling v.1.9.4 :)
## Examples and tutorials at livebook.datascienceheroes.com
##  / Now in Spanish: librovivodecienciadedatos.ai
\end{verbatim}

\begin{Shaded}
\begin{Highlighting}[]
\KeywordTok{library}\NormalTok{(tidyverse) }
\end{Highlighting}
\end{Shaded}

\begin{verbatim}
## Warning: package 'tidyverse' was built under R version 4.0.3
\end{verbatim}

\begin{verbatim}
## -- Attaching packages --------------------------------------------------------------- tidyverse 1.3.0 --
\end{verbatim}

\begin{verbatim}
## v tibble  3.0.1     v dplyr   1.0.0
## v tidyr   1.1.2     v stringr 1.4.0
## v readr   1.3.1     v forcats 0.5.0
## v purrr   0.3.4
\end{verbatim}

\begin{verbatim}
## Warning: package 'tidyr' was built under R version 4.0.3
\end{verbatim}

\begin{verbatim}
## Warning: package 'stringr' was built under R version 4.0.3
\end{verbatim}

\begin{verbatim}
## -- Conflicts ------------------------------------------------------------------ tidyverse_conflicts() --
## x dplyr::filter()    masks stats::filter()
## x dplyr::lag()       masks stats::lag()
## x dplyr::src()       masks Hmisc::src()
## x dplyr::summarize() masks Hmisc::summarize()
\end{verbatim}

\begin{Shaded}
\begin{Highlighting}[]
\KeywordTok{library}\NormalTok{(Hmisc)}
\KeywordTok{library}\NormalTok{(ggplot2)}



\CommentTok{# basic_eda <- function(data)}
\CommentTok{# \{}
\CommentTok{#   glimpse(data)}
\CommentTok{#   print(status(data))}
\CommentTok{#   freq(data) }
\CommentTok{#   print(profiling_num(data))}
\CommentTok{#   plot_num(data)}
\CommentTok{#   describe(data)}
\CommentTok{# \}}
\CommentTok{# }
\CommentTok{# glimpse(train_data)}
\CommentTok{# #We can see the general summary of the training data which contains 428 rows/obversations and 9 columns. Each column except the}
\CommentTok{# #last one represents a feature (X). The data type of each feature is shown and there are 7 int type and 2 double type. }
\CommentTok{# }
\CommentTok{# status(train_data)}
\CommentTok{# #From the table shown, we can conclude the percertanges of zeros, N/A, inf. And the table shows that there are 223 0s, which correspond}
\CommentTok{# #52.1% of total outcome. The zeros represents the cases of not having diabetes}
\CommentTok{# }
\CommentTok{# freq(train_data$Outcome)#To view the percertage of the outcomes }
\CommentTok{# }
\CommentTok{# }
\CommentTok{# #To check the numeric data }
\CommentTok{# plot_num(train_data$Pregnancies)}
\CommentTok{# }
\CommentTok{# train_data_prof=profiling_num(train_data)}
\CommentTok{# #Check the mean, std_dev of each variable}
\CommentTok{# }
\CommentTok{# df = train_data[, c("Pregnancies", "Glucose", "BloodPressure", "SkinThickness", "Insulin", "BMI", "DiabetesPedigreeFunction", "Age", "Outcome")]}
\CommentTok{# pairs(df)}
\CommentTok{# #Check the relationship between each pair of two variables.}
\CommentTok{# library(GGally)}
\CommentTok{# ggpairs(df, title="correlogram with ggpairs()") }
\CommentTok{# #Check the correlations between each pair of two variables.}



\KeywordTok{library}\NormalTok{(class)}

\NormalTok{x_train =}\StringTok{ }\NormalTok{train_data[,}\DecValTok{1}\OperatorTok{:}\DecValTok{8}\NormalTok{]}
\NormalTok{y_train =}\StringTok{ }\NormalTok{train_data}\OperatorTok{$}\NormalTok{Outcome}
\NormalTok{x_test =}\StringTok{ }\NormalTok{test_data[,}\DecValTok{1}\OperatorTok{:}\DecValTok{8}\NormalTok{]}
\NormalTok{y_test =}\StringTok{ }\NormalTok{test_data}\OperatorTok{$}\NormalTok{Outcome}

\NormalTok{scaled_x_train =}\StringTok{ }\KeywordTok{scale}\NormalTok{(x_train)}
\NormalTok{scaled_x_test =}\StringTok{ }\KeywordTok{scale}\NormalTok{(x_test)}




\NormalTok{k =}\StringTok{ }\DecValTok{1}\OperatorTok{:}\DecValTok{50}
\NormalTok{test_error <-}\StringTok{ }\KeywordTok{rep}\NormalTok{(}\DataTypeTok{x =} \DecValTok{0}\NormalTok{, }\DataTypeTok{times =} \KeywordTok{length}\NormalTok{(k))}
\NormalTok{train_error <-}\StringTok{ }\KeywordTok{rep}\NormalTok{(}\DataTypeTok{x =} \DecValTok{0}\NormalTok{, }\DataTypeTok{times =} \KeywordTok{length}\NormalTok{(k))}

\ControlFlowTok{for}\NormalTok{ (i }\ControlFlowTok{in} \KeywordTok{seq_along}\NormalTok{(k)) \{}
\NormalTok{  pred <-}\StringTok{ }\KeywordTok{knn}\NormalTok{(}\DataTypeTok{train =}\NormalTok{ scaled_x_train, }\DataTypeTok{test  =}\NormalTok{ scaled_x_test, }\DataTypeTok{cl =}\NormalTok{ y_train, }\DataTypeTok{k =}\NormalTok{ k[i])}
\NormalTok{  test_error[i] <-}\StringTok{ }\KeywordTok{mean}\NormalTok{(y_test }\OperatorTok{!=}\StringTok{ }\NormalTok{pred)}
\NormalTok{\}}

\ControlFlowTok{for}\NormalTok{ (j }\ControlFlowTok{in} \KeywordTok{seq_along}\NormalTok{(k))\{}
\NormalTok{  pred <-}\StringTok{ }\KeywordTok{knn}\NormalTok{(}\DataTypeTok{train=}\NormalTok{scaled_x_train, }\DataTypeTok{test =}\NormalTok{ scaled_x_train, }\DataTypeTok{cl =}\NormalTok{ y_train, }\DataTypeTok{k =}\NormalTok{ k[j])}
\NormalTok{  train_error[j] <-}\StringTok{ }\KeywordTok{mean}\NormalTok{(y_train }\OperatorTok{!=}\StringTok{ }\NormalTok{pred)}
\NormalTok{\}}
\NormalTok{complexity =}\StringTok{ }\KeywordTok{rep}\NormalTok{(}\DataTypeTok{x =} \DecValTok{0}\NormalTok{, }\DataTypeTok{times =} \KeywordTok{length}\NormalTok{(k))}
\ControlFlowTok{for}\NormalTok{ (i }\ControlFlowTok{in} \KeywordTok{seq_along}\NormalTok{(k))\{}
\NormalTok{  complexity[i] =}\StringTok{ }\DecValTok{1}\OperatorTok{/}\NormalTok{i}
\NormalTok{\}}


\KeywordTok{matplot}\NormalTok{(}\KeywordTok{cbind}\NormalTok{(test_error,train_error),}\DataTypeTok{type=}\StringTok{"b"}\NormalTok{,}\DataTypeTok{col=}\KeywordTok{c}\NormalTok{(}\StringTok{"red"}\NormalTok{,}\StringTok{"green"}\NormalTok{),}\DataTypeTok{lty=}\KeywordTok{c}\NormalTok{(}\DecValTok{1}\NormalTok{,}\DecValTok{1}\NormalTok{))}
\end{Highlighting}
\end{Shaded}

\includegraphics{Q4_files/figure-latex/Q4-1.pdf}

\end{document}
